\documentclass[]{article}
\usepackage{lmodern}
\usepackage{amssymb,amsmath}
\usepackage{ifxetex,ifluatex}
\usepackage{fixltx2e} % provides \textsubscript
\ifnum 0\ifxetex 1\fi\ifluatex 1\fi=0 % if pdftex
  \usepackage[T1]{fontenc}
  \usepackage[utf8]{inputenc}
\else % if luatex or xelatex
  \ifxetex
    \usepackage{mathspec}
  \else
    \usepackage{fontspec}
  \fi
  \defaultfontfeatures{Ligatures=TeX,Scale=MatchLowercase}
\fi
% use upquote if available, for straight quotes in verbatim environments
\IfFileExists{upquote.sty}{\usepackage{upquote}}{}
% use microtype if available
\IfFileExists{microtype.sty}{%
\usepackage{microtype}
\UseMicrotypeSet[protrusion]{basicmath} % disable protrusion for tt fonts
}{}
\usepackage[margin=1in]{geometry}
\usepackage{hyperref}
\hypersetup{unicode=true,
            pdftitle={Activity\_Monitoring\_Report},
            pdfauthor={Ravi M.B},
            pdfborder={0 0 0},
            breaklinks=true}
\urlstyle{same}  % don't use monospace font for urls
\usepackage{color}
\usepackage{fancyvrb}
\newcommand{\VerbBar}{|}
\newcommand{\VERB}{\Verb[commandchars=\\\{\}]}
\DefineVerbatimEnvironment{Highlighting}{Verbatim}{commandchars=\\\{\}}
% Add ',fontsize=\small' for more characters per line
\usepackage{framed}
\definecolor{shadecolor}{RGB}{248,248,248}
\newenvironment{Shaded}{\begin{snugshade}}{\end{snugshade}}
\newcommand{\KeywordTok}[1]{\textcolor[rgb]{0.13,0.29,0.53}{\textbf{#1}}}
\newcommand{\DataTypeTok}[1]{\textcolor[rgb]{0.13,0.29,0.53}{#1}}
\newcommand{\DecValTok}[1]{\textcolor[rgb]{0.00,0.00,0.81}{#1}}
\newcommand{\BaseNTok}[1]{\textcolor[rgb]{0.00,0.00,0.81}{#1}}
\newcommand{\FloatTok}[1]{\textcolor[rgb]{0.00,0.00,0.81}{#1}}
\newcommand{\ConstantTok}[1]{\textcolor[rgb]{0.00,0.00,0.00}{#1}}
\newcommand{\CharTok}[1]{\textcolor[rgb]{0.31,0.60,0.02}{#1}}
\newcommand{\SpecialCharTok}[1]{\textcolor[rgb]{0.00,0.00,0.00}{#1}}
\newcommand{\StringTok}[1]{\textcolor[rgb]{0.31,0.60,0.02}{#1}}
\newcommand{\VerbatimStringTok}[1]{\textcolor[rgb]{0.31,0.60,0.02}{#1}}
\newcommand{\SpecialStringTok}[1]{\textcolor[rgb]{0.31,0.60,0.02}{#1}}
\newcommand{\ImportTok}[1]{#1}
\newcommand{\CommentTok}[1]{\textcolor[rgb]{0.56,0.35,0.01}{\textit{#1}}}
\newcommand{\DocumentationTok}[1]{\textcolor[rgb]{0.56,0.35,0.01}{\textbf{\textit{#1}}}}
\newcommand{\AnnotationTok}[1]{\textcolor[rgb]{0.56,0.35,0.01}{\textbf{\textit{#1}}}}
\newcommand{\CommentVarTok}[1]{\textcolor[rgb]{0.56,0.35,0.01}{\textbf{\textit{#1}}}}
\newcommand{\OtherTok}[1]{\textcolor[rgb]{0.56,0.35,0.01}{#1}}
\newcommand{\FunctionTok}[1]{\textcolor[rgb]{0.00,0.00,0.00}{#1}}
\newcommand{\VariableTok}[1]{\textcolor[rgb]{0.00,0.00,0.00}{#1}}
\newcommand{\ControlFlowTok}[1]{\textcolor[rgb]{0.13,0.29,0.53}{\textbf{#1}}}
\newcommand{\OperatorTok}[1]{\textcolor[rgb]{0.81,0.36,0.00}{\textbf{#1}}}
\newcommand{\BuiltInTok}[1]{#1}
\newcommand{\ExtensionTok}[1]{#1}
\newcommand{\PreprocessorTok}[1]{\textcolor[rgb]{0.56,0.35,0.01}{\textit{#1}}}
\newcommand{\AttributeTok}[1]{\textcolor[rgb]{0.77,0.63,0.00}{#1}}
\newcommand{\RegionMarkerTok}[1]{#1}
\newcommand{\InformationTok}[1]{\textcolor[rgb]{0.56,0.35,0.01}{\textbf{\textit{#1}}}}
\newcommand{\WarningTok}[1]{\textcolor[rgb]{0.56,0.35,0.01}{\textbf{\textit{#1}}}}
\newcommand{\AlertTok}[1]{\textcolor[rgb]{0.94,0.16,0.16}{#1}}
\newcommand{\ErrorTok}[1]{\textcolor[rgb]{0.64,0.00,0.00}{\textbf{#1}}}
\newcommand{\NormalTok}[1]{#1}
\usepackage{graphicx,grffile}
\makeatletter
\def\maxwidth{\ifdim\Gin@nat@width>\linewidth\linewidth\else\Gin@nat@width\fi}
\def\maxheight{\ifdim\Gin@nat@height>\textheight\textheight\else\Gin@nat@height\fi}
\makeatother
% Scale images if necessary, so that they will not overflow the page
% margins by default, and it is still possible to overwrite the defaults
% using explicit options in \includegraphics[width, height, ...]{}
\setkeys{Gin}{width=\maxwidth,height=\maxheight,keepaspectratio}
\IfFileExists{parskip.sty}{%
\usepackage{parskip}
}{% else
\setlength{\parindent}{0pt}
\setlength{\parskip}{6pt plus 2pt minus 1pt}
}
\setlength{\emergencystretch}{3em}  % prevent overfull lines
\providecommand{\tightlist}{%
  \setlength{\itemsep}{0pt}\setlength{\parskip}{0pt}}
\setcounter{secnumdepth}{0}
% Redefines (sub)paragraphs to behave more like sections
\ifx\paragraph\undefined\else
\let\oldparagraph\paragraph
\renewcommand{\paragraph}[1]{\oldparagraph{#1}\mbox{}}
\fi
\ifx\subparagraph\undefined\else
\let\oldsubparagraph\subparagraph
\renewcommand{\subparagraph}[1]{\oldsubparagraph{#1}\mbox{}}
\fi

%%% Use protect on footnotes to avoid problems with footnotes in titles
\let\rmarkdownfootnote\footnote%
\def\footnote{\protect\rmarkdownfootnote}

%%% Change title format to be more compact
\usepackage{titling}

% Create subtitle command for use in maketitle
\newcommand{\subtitle}[1]{
  \posttitle{
    \begin{center}\large#1\end{center}
    }
}

\setlength{\droptitle}{-2em}
  \title{Activity\_Monitoring\_Report}
  \pretitle{\vspace{\droptitle}\centering\huge}
  \posttitle{\par}
  \author{Ravi M.B}
  \preauthor{\centering\large\emph}
  \postauthor{\par}
  \predate{\centering\large\emph}
  \postdate{\par}
  \date{April 1, 2018}


\begin{document}
\maketitle

\section{Introduction}\label{introduction}

This report is written for elucidating the R script
\textbf{\emph{activity\_data\_anlysis.R}} written for \emph{Course
Project 1 of Reproducible Research}, the R script
\textbf{\emph{activity\_data\_analysis.R}} analyses the data from an
anonymous individual personal activity monitoring device, this device as
sugested on
\href{https://www.coursera.org/learn/reproducible-research/peer/gYyPt/course-project-1}{CoursEra
webpage} collects data at 5 minute intervals through out the day.

The data consists of two months of data collected during the months of
October and November, 2012 and include the number of steps taken in 5
minute intervals each day.

To finish the project,I have categorized the project in to tasks listed
on CoursEra webpage they are described below

\subsection{Goal 1: Loading and Processing
data}\label{goal-1-loading-and-processing-data}

Activity Data set is extracted from
\href{https://d396qusza40orc.cloudfront.net/repdata\%2Fdata\%2Factivity.zip}{CourseEra
course project 1 instruction page}, post extracting of data I have set
the working directory followed by unzipping the file to
``step\_data.csv''. The data is then loaded and stro under variable
\textbf{\emph{``read\_activity''}}.

Variable names and the structure of the file are given by

\begin{Shaded}
\begin{Highlighting}[]
\KeywordTok{setwd}\NormalTok{(}\StringTok{"~/R/ExploratoryAnalysis/Week2/CourseProject"}\NormalTok{)}
\NormalTok{data_url <-}\StringTok{ "https://d396qusza40orc.cloudfront.net/repdata%2Fdata%2Factivity.zip"}
\NormalTok{data_activity <-}\StringTok{ "step_data.zip"}
\KeywordTok{download.file}\NormalTok{(data_url, data_activity)}
\KeywordTok{unzip}\NormalTok{(data_activity)}
\NormalTok{read_activity <-}\StringTok{ }\KeywordTok{read.csv}\NormalTok{(}\StringTok{"activity.csv"}\NormalTok{, }\DataTypeTok{sep =} \StringTok{","}\NormalTok{)}
\KeywordTok{dim}\NormalTok{(read_activity)}
\end{Highlighting}
\end{Shaded}

\begin{verbatim}
## [1] 17568     3
\end{verbatim}

\begin{Shaded}
\begin{Highlighting}[]
\KeywordTok{names}\NormalTok{(read_activity)}
\end{Highlighting}
\end{Shaded}

\begin{verbatim}
## [1] "steps"    "date"     "interval"
\end{verbatim}

\begin{Shaded}
\begin{Highlighting}[]
\KeywordTok{head}\NormalTok{(read_activity)}
\end{Highlighting}
\end{Shaded}

\begin{verbatim}
##   steps       date interval
## 1    NA 2012-10-01        0
## 2    NA 2012-10-01        5
## 3    NA 2012-10-01       10
## 4    NA 2012-10-01       15
## 5    NA 2012-10-01       20
## 6    NA 2012-10-01       25
\end{verbatim}

\begin{Shaded}
\begin{Highlighting}[]
\KeywordTok{tail}\NormalTok{(read_activity)}
\end{Highlighting}
\end{Shaded}

\begin{verbatim}
##       steps       date interval
## 17563    NA 2012-11-30     2330
## 17564    NA 2012-11-30     2335
## 17565    NA 2012-11-30     2340
## 17566    NA 2012-11-30     2345
## 17567    NA 2012-11-30     2350
## 17568    NA 2012-11-30     2355
\end{verbatim}

\begin{Shaded}
\begin{Highlighting}[]
\KeywordTok{str}\NormalTok{(read_activity)}
\end{Highlighting}
\end{Shaded}

\begin{verbatim}
## 'data.frame':    17568 obs. of  3 variables:
##  $ steps   : int  NA NA NA NA NA NA NA NA NA NA ...
##  $ date    : Factor w/ 61 levels "2012-10-01","2012-10-02",..: 1 1 1 1 1 1 1 1 1 1 ...
##  $ interval: int  0 5 10 15 20 25 30 35 40 45 ...
\end{verbatim}

after loding and processing the data missing values are ignored and mean
of the total number of steps taken per day is evaluated as follows

\begin{Shaded}
\begin{Highlighting}[]
\NormalTok{read_activity}\OperatorTok{$}\NormalTok{date <-}\StringTok{ }\KeywordTok{as.Date}\NormalTok{(read_activity}\OperatorTok{$}\NormalTok{date)}
\NormalTok{clear_na_data <-}\StringTok{ }\KeywordTok{subset}\NormalTok{(read_activity, }\OperatorTok{!}\KeywordTok{is.na}\NormalTok{(read_activity}\OperatorTok{$}\NormalTok{steps))}
\NormalTok{total_steps_sum <-}
\StringTok{  }\KeywordTok{tapply}\NormalTok{(}
\NormalTok{    clear_na_data}\OperatorTok{$}\NormalTok{steps,}
\NormalTok{    clear_na_data}\OperatorTok{$}\NormalTok{date,}
\NormalTok{    sum,}
    \DataTypeTok{na.rm =} \OtherTok{TRUE}\NormalTok{,}
    \DataTypeTok{simplify =}\NormalTok{ T}
\NormalTok{  )}
\end{Highlighting}
\end{Shaded}

histogram of the total number of steps taken each day is plotted using
the below R script and also mean and median are calculated

\begin{Shaded}
\begin{Highlighting}[]
\KeywordTok{hist}\NormalTok{(}\DataTypeTok{x=}\NormalTok{total_steps_sum,}
     \DataTypeTok{col=}\StringTok{"red"}\NormalTok{,}
     \DataTypeTok{breaks=}\DecValTok{10}\NormalTok{,}
     \DataTypeTok{xlab=}\StringTok{"Daily steps"}\NormalTok{,}
     \DataTypeTok{ylab=}\StringTok{"Frequency"}\NormalTok{,}
     \DataTypeTok{main=}\StringTok{"Total Number of Steps per Day neglecting missing data"}\NormalTok{)}
\end{Highlighting}
\end{Shaded}

\includegraphics{Activity_Monitoring_Report_files/figure-latex/unnamed-chunk-3-1.pdf}

\begin{Shaded}
\begin{Highlighting}[]
\KeywordTok{mean}\NormalTok{(total_steps_sum)}
\end{Highlighting}
\end{Shaded}

\begin{verbatim}
## [1] 10766.19
\end{verbatim}

\begin{Shaded}
\begin{Highlighting}[]
\KeywordTok{median}\NormalTok{(total_steps_sum)}
\end{Highlighting}
\end{Shaded}

\begin{verbatim}
## [1] 10765
\end{verbatim}

\subsection{Goal 2: Time series plot of Average daily activity
pattern}\label{goal-2-time-series-plot-of-average-daily-activity-pattern}

I have costructed time series plot ** i.e type = \emph{``l''} ** of the
5-minute interval (x-axis) and the average number of steps taken,
averaged across all days (y-axis) also evaluated Which 5-minute
interval, on average across all the days in the dataset, contains the
maximum number of steps

\begin{Shaded}
\begin{Highlighting}[]
\NormalTok{avg_daily <-}\StringTok{ }\KeywordTok{tapply}\NormalTok{(clear_na_data}\OperatorTok{$}\NormalTok{steps, clear_na_data}\OperatorTok{$}\NormalTok{interval, mean, }\DataTypeTok{na.rm=}\OtherTok{TRUE}\NormalTok{, }\DataTypeTok{simplify=}\NormalTok{T)}
\NormalTok{avg_daily_interval <-}\StringTok{ }\KeywordTok{data.frame}\NormalTok{(}\DataTypeTok{interval=}\KeywordTok{as.integer}\NormalTok{(}\KeywordTok{names}\NormalTok{(avg_daily)), }\DataTypeTok{avg=}\NormalTok{avg_daily)}
\KeywordTok{with}\NormalTok{(avg_daily_interval,}
     \KeywordTok{plot}\NormalTok{(interval,}
\NormalTok{          avg,}
          \DataTypeTok{type=}\StringTok{"l"}\NormalTok{,}
          \DataTypeTok{xlab=}\StringTok{"5-minute intervals"}\NormalTok{,}
          \DataTypeTok{ylab=}\StringTok{"average number of steps"}\NormalTok{))}
\end{Highlighting}
\end{Shaded}

\includegraphics{Activity_Monitoring_Report_files/figure-latex/unnamed-chunk-4-1.pdf}

\begin{Shaded}
\begin{Highlighting}[]
\NormalTok{max_num_steps <-}\StringTok{ }\KeywordTok{max}\NormalTok{(avg_daily_interval}\OperatorTok{$}\NormalTok{avg)}
\NormalTok{avg_daily_interval[avg_daily_interval}\OperatorTok{$}\NormalTok{avg }\OperatorTok{==}\StringTok{ }\NormalTok{max_num_steps, ]}
\end{Highlighting}
\end{Shaded}

\begin{verbatim}
##     interval      avg
## 835      835 206.1698
\end{verbatim}

here \textbf{\emph{avg\_daily\_interval}} and **\_max\_num\_steps**
variables were created to store the relevant results.

\subsection{Goal 3: Imputing missing
values}\label{goal-3-imputing-missing-values}

Further a new data set which would consider or impute the missing values
is costructed and the resulting data set evaluations are compared with
the results where the missing values were neglected, the data from both
the sets (missing values neglected and missing values imputed ) when
compared it cold be inferred that mean and median were almost similar,
please check the \textbf{ImputeData\_MissingData.png} file

\begin{Shaded}
\begin{Highlighting}[]
\KeywordTok{nrow}\NormalTok{(read_activity[}\KeywordTok{is.na}\NormalTok{(read_activity}\OperatorTok{$}\NormalTok{steps),])}
\end{Highlighting}
\end{Shaded}

\begin{verbatim}
## [1] 2304
\end{verbatim}

\begin{Shaded}
\begin{Highlighting}[]
\NormalTok{impute_miss <-}\StringTok{ }\NormalTok{read_activity}
\NormalTok{na_data <-}\StringTok{ }\KeywordTok{is.na}\NormalTok{(impute_miss}\OperatorTok{$}\NormalTok{steps)}
\NormalTok{clear_na_avg <-}\StringTok{ }\KeywordTok{tapply}\NormalTok{(clear_na_data}\OperatorTok{$}\NormalTok{steps, clear_na_data}\OperatorTok{$}\NormalTok{interval, mean, }\DataTypeTok{na.rm=}\OtherTok{TRUE}\NormalTok{, }\DataTypeTok{simplify=}\NormalTok{T)}
\NormalTok{impute_miss}\OperatorTok{$}\NormalTok{steps[na_data] <-}\StringTok{ }\NormalTok{clear_na_avg[}\KeywordTok{as.character}\NormalTok{(impute_miss}\OperatorTok{$}\NormalTok{interval[na_data])]}
\NormalTok{new_data <-}\StringTok{ }\KeywordTok{tapply}\NormalTok{(impute_miss}\OperatorTok{$}\NormalTok{steps, impute_miss}\OperatorTok{$}\NormalTok{date, sum, }\DataTypeTok{na.rm=}\OtherTok{TRUE}\NormalTok{, }\DataTypeTok{simplify=}\NormalTok{T)}

\KeywordTok{hist}\NormalTok{(}\DataTypeTok{x=}\NormalTok{new_data,}
     \DataTypeTok{col=}\StringTok{"blue"}\NormalTok{,}
     \DataTypeTok{breaks=}\DecValTok{10}\NormalTok{,}
     \DataTypeTok{xlab=}\StringTok{"Daily steps"}\NormalTok{,}
     \DataTypeTok{ylab=}\StringTok{"Frequency"}\NormalTok{,}
     \DataTypeTok{main=}\StringTok{"Total number of steps(comparison with missing data imputed)"}\NormalTok{)}


\KeywordTok{hist}\NormalTok{(}\DataTypeTok{x=}\NormalTok{total_steps_sum,}
     \DataTypeTok{col=}\StringTok{"red"}\NormalTok{,}
     \DataTypeTok{breaks=}\DecValTok{10}\NormalTok{,}
     \DataTypeTok{xlab=}\StringTok{"Daily steps"}\NormalTok{,}
     \DataTypeTok{ylab=}\StringTok{"Frequency"}\NormalTok{,}
     \DataTypeTok{main=}\StringTok{"Total number of steps(comparison with missing data imputed)"}\NormalTok{,}
     \DataTypeTok{add=}\NormalTok{T)}
\KeywordTok{legend}\NormalTok{(}\StringTok{"topright"}\NormalTok{, }\KeywordTok{c}\NormalTok{(}\StringTok{"Imputed Data"}\NormalTok{, }\StringTok{"Non-NA Data"}\NormalTok{), }\DataTypeTok{fill=}\KeywordTok{c}\NormalTok{(}\StringTok{"blue"}\NormalTok{, }\StringTok{"red"}\NormalTok{) )}
\end{Highlighting}
\end{Shaded}

\includegraphics{Activity_Monitoring_Report_files/figure-latex/unnamed-chunk-5-1.pdf}

\begin{Shaded}
\begin{Highlighting}[]
\KeywordTok{mean}\NormalTok{(new_data)}
\end{Highlighting}
\end{Shaded}

\begin{verbatim}
## [1] 10766.19
\end{verbatim}

\begin{Shaded}
\begin{Highlighting}[]
\KeywordTok{median}\NormalTok{(new_data)}
\end{Highlighting}
\end{Shaded}

\begin{verbatim}
## [1] 10766.19
\end{verbatim}

\subsection{Goal 4: Evaluating weekdays and weekend
patterns}\label{goal-4-evaluating-weekdays-and-weekend-patterns}

I have Created a new factor variable \textbf{check\_day} in the dataset
with two levels - ``weekday'' and ``weekend'' indicating whether a given
date is a weekday or weekend day.

\begin{Shaded}
\begin{Highlighting}[]
\NormalTok{check_day <-}\StringTok{ }\ControlFlowTok{function}\NormalTok{(days) \{}
\NormalTok{  wd <-}\StringTok{ }\KeywordTok{weekdays}\NormalTok{(days)}
  \KeywordTok{ifelse}\NormalTok{ (wd }\OperatorTok{==}\StringTok{ "Saturday"} \OperatorTok{|}\StringTok{ }\NormalTok{wd }\OperatorTok{==}\StringTok{ "Sunday"}\NormalTok{, }\StringTok{"weekend"}\NormalTok{, }\StringTok{"weekday"}\NormalTok{)}
\NormalTok{\}}

\NormalTok{test_data <-}\StringTok{ }\KeywordTok{sapply}\NormalTok{(impute_miss}\OperatorTok{$}\NormalTok{date, check_day)}
\NormalTok{impute_miss}\OperatorTok{$}\NormalTok{week <-}\StringTok{ }\KeywordTok{as.factor}\NormalTok{(test_data)}
\KeywordTok{dim}\NormalTok{(impute_miss)}
\end{Highlighting}
\end{Shaded}

\begin{verbatim}
## [1] 17568     4
\end{verbatim}

\begin{Shaded}
\begin{Highlighting}[]
\KeywordTok{head}\NormalTok{(impute_miss)}
\end{Highlighting}
\end{Shaded}

\begin{verbatim}
##       steps       date interval    week
## 1 1.7169811 2012-10-01        0 weekday
## 2 0.3396226 2012-10-01        5 weekday
## 3 0.1320755 2012-10-01       10 weekday
## 4 0.1509434 2012-10-01       15 weekday
## 5 0.0754717 2012-10-01       20 weekday
## 6 2.0943396 2012-10-01       25 weekday
\end{verbatim}

\begin{Shaded}
\begin{Highlighting}[]
\KeywordTok{tail}\NormalTok{(impute_miss)}
\end{Highlighting}
\end{Shaded}

\begin{verbatim}
##           steps       date interval    week
## 17563 2.6037736 2012-11-30     2330 weekday
## 17564 4.6981132 2012-11-30     2335 weekday
## 17565 3.3018868 2012-11-30     2340 weekday
## 17566 0.6415094 2012-11-30     2345 weekday
## 17567 0.2264151 2012-11-30     2350 weekday
## 17568 1.0754717 2012-11-30     2355 weekday
\end{verbatim}

\begin{Shaded}
\begin{Highlighting}[]
\NormalTok{week_data <-}\StringTok{ }\KeywordTok{aggregate}\NormalTok{(steps }\OperatorTok{~}\StringTok{ }\NormalTok{week}\OperatorTok{+}\NormalTok{interval, }\DataTypeTok{data=}\NormalTok{impute_miss, }\DataTypeTok{FUN=}\NormalTok{mean)}
\end{Highlighting}
\end{Shaded}

the resulting \emph{time series plot} with \emph{type=``1''} of the
5-minute interval for the average number of steps taken and average
across all weekday days or weekend days is plotted as follows

\begin{Shaded}
\begin{Highlighting}[]
\KeywordTok{library}\NormalTok{(lattice)}
\KeywordTok{xyplot}\NormalTok{(steps }\OperatorTok{~}\StringTok{ }\NormalTok{interval }\OperatorTok{|}\StringTok{ }\KeywordTok{factor}\NormalTok{(week),}
       \DataTypeTok{layout =} \KeywordTok{c}\NormalTok{(}\DecValTok{1}\NormalTok{, }\DecValTok{2}\NormalTok{),}
       \DataTypeTok{xlab=}\StringTok{"Interval"}\NormalTok{,}
       \DataTypeTok{ylab=}\StringTok{"Number of steps"}\NormalTok{,}
       \DataTypeTok{type=}\StringTok{"l"}\NormalTok{,}
       \DataTypeTok{lty=}\DecValTok{1}\NormalTok{,}
       \DataTypeTok{data=}\NormalTok{week_data)}
\end{Highlighting}
\end{Shaded}

\includegraphics{Activity_Monitoring_Report_files/figure-latex/unnamed-chunk-7-1.pdf}

here \textbf{week\_data} variable stores the average or aggregate of
steps taken in refernce to wekdays and wekends.

\subsection{Summary}\label{summary}

This project report meets all the below mentioned criteria for the
submission

\begin{enumerate}
\def\labelenumi{\arabic{enumi}.}
\tightlist
\item
  Code for reading in the dataset and/or processing the data
\item
  Histogram of the total number of steps taken each day
\item
  Mean and median number of steps taken each day
\item
  Time series plot of the average number of steps taken
\item
  The 5-minute interval that, on average, contains the maximum number of
  steps
\item
  Code to describe and show a strategy for imputing missing data
\item
  Histogram of the total number of steps taken each day after missing
  values are imputed
\item
  Panel plot comparing the average number of steps taken per 5-minute
  interval across weekdays and weekends
\item
  All the R code needed to reproduce the results (numbers, plots, etc.)
  in the report
\end{enumerate}


\end{document}
